\begin{DoxyAuthor}{Author}
Rodrigues Filho \href{mailto:edno-moura@mail.com}{\tt edno-\/moura@mail.\+com}
\end{DoxyAuthor}
\hypertarget{index_what_is}{}\section{What is Cmd\+Messenger?}\label{index_what_is}
The Cmd\+Messenger is a messaging library for the Arduino platform. To use the Cmd\+Messenger, a list of command identifiers is defined and then, callbacks are attached for the received messages. The message format is as it follows\+: 
\begin{DoxyPre}
  cmdID,arg1,arg2,...,argn;
\end{DoxyPre}
\hypertarget{index_this_project}{}\section{What is this project all about?}\label{index_this_project}
This is simply an implementation of such messaging protocol for the pc side, built in a cross-\/platform C++. It uses the Serial Library for its communication, this library provides total control over Timeouts and it is cross-\/platform. The official project of the Cmd\+Messenger provides a full C\# implementation that runs in both Mono and Visual Studio.\hypertarget{index_getting_started}{}\section{Getting Started}\label{index_getting_started}
Take a look at the main class documentation \hyperlink{classcmd_1_1_cmd_messenger}{cmd\+::\+Cmd\+Messenger}

\begin{DoxyRefDesc}{Todo}
\item[\hyperlink{todo__todo000001}{Todo}]write the getting started section 

Remember to find a way to ship all dependecies in one package \end{DoxyRefDesc}
\hypertarget{index_features}{}\section{Features}\label{index_features}

\begin{DoxyItemize}
\item Total control over Timeouts trough the Serial Library project.
\item Send and receive ascii commands.
\item Clean A\+P\+I with overloaded operators.
\end{DoxyItemize}\hypertarget{index_problems}{}\section{Problems}\label{index_problems}

\begin{DoxyItemize}
\item Escaping not yet implemented.
\item Does not send or receive binary data.
\item Parsers safety and efficiency needs to be improved.
\item Error check, and loss of data problems.
\item It's written in cross-\/platform C++, but the project lacks support for other platforms rather than linux.
\end{DoxyItemize}\hypertarget{index_install}{}\section{Installation}\label{index_install}
\begin{DoxyRefDesc}{Todo}
\item[\hyperlink{todo__todo000002}{Todo}]Write the install section \end{DoxyRefDesc}
